%\textbf{SUMÁRIO}

%Prefácio de 1960

%Semente de crápula (os 134 aforismos)~

%Prefácio inédito de 1955: Semente de crápula ou o charlatão de boa vontade. Autocrítica de um educador

%Sobre a tradução

\pagebreak
\thispagestyle{empty}
\movetooddpage
\thispagestyle{empty}
\setcounter{footnote}{0}
\begin{vplace}[0.25]


{\large\Formular{
\noindent{}Sobre a tradução\footnote{Parte deste e do texto da contracapa
  foi escrita com base em textos de Sandra Alvarez de Toledo presentes
  nos livros {\slsc{Œuvres}} (L'Arachnéen, 2007) e {\slsc{Permitir, trazar,
    ver}} (\scalebox{.8}{MACBA}, 2009).}
}}
\end{vplace}

\pagebreak
\thispagestyle{empty}

\movetooddpage

Os aforismos de \emph{Semente de crápula. Conselhos aos educadores que
queiram cultivá"-la} são o resultado das primeiras tentativas de Deligny.
Ele passa os primeiros 20 anos de atuação profissional entre escolas
especiais, instituições médico"-pedagógicas e hospitais psiquiátricos.
Durante a Segunda Guerra, une operários e membros da Resistência em uma
rede de ajuda mútua, alojada em edifícios destruídos ou abandonados nos
bairros populares. Aproxima, assim, pela prática, educadores diplomados,
de origem pequeno"-burguesa, do meio social dos delinquentes.

\emph{Semente de crápula} foi escrito em 1943 e publicado em 1945, numa
edição que esgota rapidamente.

Por ocasião de um primeiro projeto de reedição em 1955, Deligny redige
um prefácio intitulado ``Semente de crápula ou o charlatão de boa
vontade. Autocrítica de um educador''. Mas a reedição, que só aparece em
1960, é publicada com um outro prefácio no qual o autor vislumbra novo
título para o livro: ``Semente de crápula ou o amador de pipas'' (a
palavra ``amateur'' ressoa também com ``amante''). ``Por volta de 1943, eu
me pus a fazer uma pipa, duas pipas: as fórmulas, formulinhas, cantigas,
charadas, aforismos e paradoxos de \emph{Semente de crápula}.'' Deligny
havia iniciado, uns anos antes da reedição, sua conhecida
tentativa/experimentação/invenção com autistas, que durará até sua morte
em 1996. Deligny acolhe autistas sem intenção de curá"-los, e fala do
autismo sem ser psiquiatra.

O projeto de nossa tradução dos aforismos de Se\emph{mente de crápula}
origina"-se em uma prática. A partir da lógica do amante/amador temos
traduzido textos diversos, de autores consagrados a desconhecidos,
dentro de uma ação chamada \emph{Ensaios ignorantes}, que realizamos nos
últimos 9 anos. Juliana conheceu Deligny em 2012 vendo filmes, mapas,
traços, como visitante numa mostra de arte, e ficou, desde então, entre
textos dele no original francês e em tradução espanhola (não havia nada
traduzido em português). Três anos depois, propôs leituras coletivas de
alguns textos de Deligny e foi aí que Luiz se aproximou. Começamos a ler
entre francês e espanhol, vertendo já para nossa língua, de um jeito
\emph{ignorante} (em aliança com Joseph Jacotot): por comparação.
Podemos ler, traduzir, entender e dar a ler. Em 2019, os \emph{Ensaios
ignorantes} foram convidados pela curadoria de Artes Visuais para criar
algo dentro de um festival no complexo hospitalar do Juquery. Decidimos
traduzir os 134 aforismos do livro para ocupar o espaço que foi a
recepção da antiga colônia psiquiátrica de Franco da Rocha, e ler os
\emph{conselhos} junto com o público. Havia, então, um objetivo de
migrar o texto do papel para o espaço e para a voz. Nenhum problema:
eram conselhos. Porém, fixar essa tradução em livro seria projeto mais
ousado, pois nossas traduções, até então, vinham desejando soar textos
entre a solidão e o comum de encontros públicos, para agir a partir da
proximidade com as palavras. A n"-1 se animou, pois algo de nossas
posições interessou à editora. Não foi fácil. Deligny nomeou esses
aforismos conselhos, cantigas, charadas, ou seja, algo que se pode dizer
com voz. Ainda que ele tenha sido um escritor impressionante, optamos
por manter ecos dessa voz (advinda de sua prática) nos aforismos em
português. Ele era francês, esperamos que dê para escutar sua língua.
Ele escrevia a partir da prática. Esperamos que se escute aqui algo
da~ação de Deligny. Em sua tentativa de asilo, o agir é também
desprovido de intenção. Nós dois intencionamos nos aproximar do livro,
vigilantes em relação a uma leitura"-tradução"-amadora/amante que o
escutasse de perto. Agora, na língua de chegada, esperamos que nossa
tentativa chegue. Agradecemos a Florelle D'Hoest, Maxime Godard e Sandra
Alvarez de Toledo.

\hfill{}\emph{Juliana Jardim e Luiz Pimentel}
