\pagebreak
\thispagestyle{empty}
\movetooddpage
\thispagestyle{empty}
\begin{vplace}[0.25]


{\large\Formular{
\noindent{}Prefácio publicado com\\ a reedição de 1960}}
\end{vplace}

\pagebreak
\thispagestyle{empty}

\movetooddpage

Este livrinho foi escrito em 1943, editado em 1945. Dez anos depois, me
falaram para fazer uma nova edição. Eu o reli. Indignado, comecei a
preparar uma crítica rigorosa daquelas pequenas fórmulas sob o título:
\emph{Semente de crápula ou o charlatão de boa vontade}. Essa
autocrítica, relida hoje, no inverno das Cévennes, me parece bem
excessiva, agressiva, peremptória. Ela ficará no caixa de madeira no
qual se amontoam, a cada mudança, páginas e páginas de intenções e
relatos que, talvez, sejam para mim o que as folhas que caem são para as
árvores.

No entanto, me incomoda deixar que saiam novos exemplares de
\emph{Semente de crápula} sem dizer nada. Tenho quinze ou dezesseis anos
a mais, quinze ou dezesseis anos nesse trabalho diário do qual eu falava
alegremente em 1943.

Palavras me vêm, páginas, capítulos, se me deixo levar.

Este livrinho precisa de um subtítulo que me situe, agora, em relação ao
que escrevi há quinze anos. Tenho este subtítulo: \emph{Semente de
crápula ou o amador de pipas}.

Era uma vez um amador de pipas. Vocês já sabem o que é uma pipa em
relação às nuvens, aos pássaros, aos aviões e satélites; ela não é
encontrada na natureza, é possível fazê"-la por si mesmo a partir de
modelos propostos em revistas ou folhetos, ou ainda inventar novas
formas inspiradas em pipas chinesas ancestrais, no abutre dos Andes ou
no avião"-caça Mystère \versal{IV}. Uma pipa não atravessa os muros do espaço, não
troveja nem zune, falta um não"-sei"-quê para que ela se sustente no vento
e continue alegrando, com um ponto de cor viva, o mais cinza dos céus,
ou para que ela caia e, pelo menos, não quebre nada além de sua própria
armação. À primeira vista, isso não serve para nada. Veremos.

Então, por volta de 1943, comecei a fazer uma pipa, duas pipas: as
fórmulas, formulinhas, cantigas, charadas, aforismos e paradoxos de
\emph{Semente de crápula}.

Uma pipa, sobretudo se for pequena, é fácil de segurar. Cento e trinta e
seis já é outro assunto: elas arrastam você, ainda que tenha pouco
vento, levantariam você, não podemos dizer que acima de você mesmo e,
contudo, acabei sendo um educador de renome, levado, pela força e pela
graça dessas cento e trinta e seis pequenas pipas, a um congresso
internacional aqui, a uma comissão ali, e por mais que eu puxasse as
cordas, como fazem os mergulhadores quando querem voltar a subir, minhas
pipas frequentemente me deixaram mofando ali, de onde eu teria querido
escapar.

Aconteceram"-me coisas piores. Sempre provocado por esse rebanho de
discursos díspares em cuja forma eu mesmo havia mexido à vontade,
encontrei"-me à frente na criação de organismos de reeducação. Pobre de
mim: aí é que se enredam os discursos e seus fios. É aí que o pobre
diabo que segura com a mão direita seu ramo de pequenas bandeiras
multiformes e multicolores se dá conta de que só lhe resta uma das mãos,
a outra, para lutar a todo custo, sem muralha nem certeza para se
apoiar. Com sorte, uma ou outra dessas fórmulas, soltas há muito tempo,
não acaba caindo sobre sua cabeça e seus ombros, cegando"-o, envolvendo"-o
com as tiras de sua rabiola, sobre as quais algumas palavras estão
escritas e ele desprega uma por acaso e a aplica e diz uma palavra em
falso como se faz um movimento em falso.

É isso provavelmente o que eu queria contar aos antigos e futuros
leitores de \emph{Semente de crápula}. Há dois mundos. Aquele das
fórmulas, formulinhas, charadas e parábolas e aquele do que acontece a
todo momento aqui embaixo com quem quer ajudar os outros. Se, uma vez
lidas, algumas de minhas palavras estremecem alegremente no céu de
algumas memórias, melhor assim: aí está sua razão de ser. Mas aquele que
quiser se servir delas, aplicá"-las de alguma maneira, se daria conta, ao
mesmo tempo, do que são feitas: pedaços de páginas lidas, coladas e
penduradas sobre os galhos maleáveis e leves arrancados de uma espécie
particular de entusiasmo que surge cada vez que um menino me aborda. Que
foi mil vezes serrado, derrubado e de cujo tronco nunca param de brotar
rebentos.

\hfill{}Fernand Deligny

\hfill{}janeiro de 1960
