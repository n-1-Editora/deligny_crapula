\movetooddpage
\thispagestyle{empty}
\setcounter{footnote}{0}
\begin{vplace}[0.25]


{\large\Formular{
\noindent{}Prefácio escrito para a\\ reedição de 1955, não\\ publicado à época}}
\end{vplace}

\pagebreak
\thispagestyle{empty}

\movetooddpage

\section{Semente de crápula ou o charlatão de boa vontade. Autocrítica de um educador}

\bigskip
\bigskip

\epigraph{Proudhon possui uma inclinação natural para a dialética. Mas, como
nunca compreendeu a verdadeira dialética científica, não foi além dos
sofismas. Na verdade, isso se explica pelo seu ponto de vista
pequeno"-burguês. O pequeno"-burguês constitui"-se de um `por uma parte'
e de um `por outra parte'\ldots{} É a contradição personificada. Se, como
Proudhon, ele é uma pessoa de espírito, logo saberá manipular suas
próprias contradições e convertê"-las, segundo as circunstâncias, em
paradoxos evidentes, espetaculares, ora escandalosos, ora brilhantes.
Charlatanismo científico e arranjos políticos são elementos inseparáveis
de semelhante ponto de vista.}{Karl Marx}

Ninguém gosta muito de ser pego de surpresa à luz de uma explicação
clara.

Escrevi \emph{Semente de crápula} por volta de 1943. As 136 pequenas
fórmulas foram editadas em 1945. Elas me valeram a simpatia de
educadores próximos e distantes.

Agora, em 1955, uma semana após outra, um pedido chega: dois
\emph{Semente de crápula} aqui, dois \emph{Semente de crápula}
acolá\ldots{} Ora, não há mais \emph{Semente de crápula}. Deve"-se
reeditá"-lo?

Por doze anos, trabalhei. Meu ofício de educador e eu não nos separamos.
A mensagem disparatada de 1943, com suas facetas talhadas em paradoxos,
me irritaria se eu a lesse vinda de um outro. Mas, já que ela ainda
flutua e que, aqui e acolá, jovens educadores ainda querem lê"-la, por
que não retomá"-la, corrigi"-la, a fim de encurtar em um dia a estrada
daqueles que querem sair dos repugnantes redemoinhos ideológicos em que
estão presos, primeiramente, aqueles e aquelas cuja tarefa cotidiana
consiste em cuidar de crianças que não nasceram deles.

``Pequeno"-burguês\ldots{}'', eis"-me entregue ao meu lugar (gostaria de poder
escrever ``meu lugar de então'') eu, o educador que se queria, ainda
assim, revolucionário.

Pequeno"-burguês\ldots{} A tribo é numerosa e muito diversa. Pequeno"-burguês
de origem? Por acaso sofri por ser um explorado? Nunca: sempre tive a
sensação de ser um privilegiado. Órfão de guerra, sim, mas órfão de
oficial, e o bairro onde cresci, em Lambersart, perto de Lille, era tão
pequeno"-burguês quanto um bairro pode ser. Ele era assim: \emph{villas},
autênticas \emph{villas}-pequenos"-castelos em cada um dos lados de uma
avenida vizinha de um hipódromo: pode"-se dizer que aquela avenida era a
espinha, a coluna vertebral do lugar. Quem vivia ali dentro? Não faço
ideia: nunca vi, com meus próprios olhos, nada entrar nem sair dali.
Havia árvores que conhecia bem e que eram minha única companhia quando
pegava aquela avenida.~Agora, quando vejo tudo aquilo com mais
distância, sei que havia um castelo, uma espécie de castelo, que se
chamava \emph{crépi} {[}reboco{]}, e eu achava que era seu nome de
construção, castelo"-reboco, como diríamos castelo"-pontudo.\footnote{A
  palavra {\slsc{crépi}} se traduz por ``reboco''; {\slsc{chapeau"-pointu}} é
  o tipo de chapéu com a forma pontiaguda. {\slsc{Chapeau}} e
  {\slsc{chateaux}} em Francês soam de modo parecido. Em seguida,
  {\slsc{crépi}} aparece como nome próprio, e, portanto, é a essa confusão
  que o autor se refere ao ouvir a palavra à época.} Na verdade, Crépi
era o sobrenome de um grande empresário têxtil, e tudo parecia indicar
que aquele imponente castelo, de tão safado, havia procriado e gerado
uma centena de \emph{villas}-castelãs, algumas retorcidas, tortas como
solteironas que andam cheias de frescura; as outras, bem burguesas,
fartas, limpas e satisfeitas em seu parque conservado, habitadas por
algum chefe ou subchefe ou notável colaboracionista; portanto, uma avenida
muito longa composta por esse enxame de \emph{villas} provenientes do
castelo, mas as últimas delas não eram mais do que casas com uma pequena
escadaria ou pequena torre ou um não"-sei"-quê na fachada ou na parte de
trás que lhes permitia aparecer ao longo dessa avenida do Hipódromo, à
espera de quê? de que um dia o castelo"-reboco passasse lentamente diante
delas?

Num extremo da avenida, um canal, o Deûle, e suas barcas amigas; no
outro, um bom e velho prostíbulo, que deve ter sido uma fazenda: um
bordel; à frente, os campos; aqui, ali, nas árvores, mais alguns ``castelos'',
mas esses não alinhados, colocados ali, em qualquer canto, e uma
estrada pavimentada que insistia em passar com um bonde que sacudia em
todas essas curvas: suas rodas mordiam o trilho, seu mastro pulava: um
bonde não foi feito para passar por um caminho que deveria ser traçado
por gaivotas preocupadas em voar ao redor da propriedade privada, em dar
voltas, em não morder, em passar por onde podiam, por onde têm permissão
de passar: caminho de costume: deve"-se respeitar os costumes.
Frequentemente acontecia de o bonde descarrilar, nervoso, sobrecarregado
por essas complicações, por essas reviravoltas, enlouquecido por essas
afetações, por esses gracejos que o obrigavam a fazer. Ele descarrilava.
Pouco importa: ele estava no fim de seu trajeto: deveria ir até a igreja
-- para que isso? Ele estava sempre mais ou menos vazio, corria, a
marcha engatada na velocidade mais alta, tomado pelo mal das
tempestades, balançava trinta vezes atrás e na frente pelo labirinto de
curvas entre os castelos secretos e parava no ``Canon d'Or'' para fazer
seu serviço, embarcar seu carregamento de uma brava gente, pessoas
recém"-saídas de suas casas de tijolos: avenidas, ruas de casinhas de
tijolos, todas iguais, construídas a cada estação do ano: era nessas
casas que viviam os empregados.

Todos ``empregados'', o bonde cheio, as ruas cheias; um bairro, uma
cidade de ``empregados''. Eles tinham a sensação de serem explorados?
Privilegiados, é isso o que eram. Havia muitos empregados de banco, e,
entre os empregados de banco, havia alguns que eram do Banco da França:
privilegiados em relação àqueles que estavam nos pequenos bancos; e os
empregados dos pequenos bancos? Privilegiados em relação àqueles que não
tinham emprego algum ou em relação ao pessoal uniformizado, um
rapaz disso
ou daquilo: ``rapaz'', com cinquenta anos? Privilegiados esses
``rapazes'', pois, em sua maior parte, eram mutilados de guerra: a
mutilação, mesmo a da guerra, era um privilégio? Em relação aos mortos,
aos moribundos, evidentemente eram privilegiados, e o brilho do
privilégio podia ser lido em muitos dos rostos: o bonde lotado voltava
para a cidade pela rua Royale: estritamente verdade.

Eu era, então, órfão e privilegiado, duplamente, triplamente
privilegiado, já que: meu pai morrera na guerra, mas ele morreu oficial:
pequena pensão; ele tinha metido na minha cabeça (ele, ou minha mãe, ou
os dois) o que era preciso para passar no exame das bolsas de estudo:
bolsista; aluno do liceu:\footnote{O liceu era e continua sendo a
  instituição escolar francesa de educação secundária. Na época de
  Deligny, este nível de ensino não era obrigatório.} privilegiado em
relação aos do primário e aos aprendizes.

Privilegiado. Não há dúvida.

Privilegiado, agarre"-se aos seus privilégios, como um acrobata nauseado
ao seu trapézio.

Os operários me pareciam sérios, e cheiravam como homens de uma outra
raça: eles tinham, no bonde, uma presença desajeitada e rude. Macacões
de trabalho, sapatos, suas roupas, suas ferramentas: sobre a plataforma
cheia de seres de pele pálida, estavam entorpecidos, muitas vezes
pensativos e, na hora de descer, se cumprimentavam com um sinal de
cabeça: no meio dessa gente, eles deviam ter medo de seu sotaque, mas eu
via ali um desdém involuntário, o desejo de não fazer ouvir suas vozes
pelos sujeitos ignóbeis que pareciam se conhecer todos entre si. Aliás,
os operários raramente pegavam o bonde, eles iam a pé, e preferiam a rua
Saint"-André à rua Royale, paralela e mais movimentada, ou a esplanada,
ao longo do canal, ou ainda as muralhas, de bicicleta.

O bonde era um privilégio que me deixava tão desconfortável quanto os
outros privilégios que eu devia suportar: órfão, inteligente, aluno do
liceu etc. ``Pequeno"-burguês'', abandone a tribo! É bem isso o que
tentei fazer, mas imagine só! a tribo, nós a temos na cabeça: e, no
entanto, fui procurar trabalho bem longe, não na direção dessas
distâncias geográficas que só me atiçaram por meio de uma imaginação
muito conscientemente gratuita, mas em pleno asilo de alienados. Um
asilo, é muito longe, é uma ilha.

Antes, quando tinha vinte anos, já havia tentado deixar a tribo. Havia
me filiado ao Partido comunista: um local vermelho, rua de Paris. Um
homem estava ali, constantemente: um membro permanente? Para mim, ele
era como o guardião de uma tradição. Era melhor não se fazer de bobo
naquele lugar. Ele e nós, não pesávamos o mesmo peso. Nós, porque éramos
quatro ou cinco estudantes usando a mesma insígnia. Ele e nós não
tínhamos a mesma densidade: tratávamos de estar à altura quando íamos
pegar os folhetos ou os cartazes para distribuir.

Os cartazes. Ali estavam, uma centena, dobrados em cima de uma mesa do
local, carcomida: havia uma tapeçaria rosa empoeirada pelo pó de gesso
que saía por entre os rasgos: parecia as bochechas maquiadas de uma
velha: os cartazes, eu devia pensar que com um cartaz (mas qual?, quais
teriam sido sua cor e seu desenho e suas palavras? Eu não sabia), um
determinado cartaz que o Partido deveria ter encontrado que, uma vez com esse cartaz colado
por toda parte, a cidade, algumas horas mais tarde, teria entrado
em comoção, todo mundo em marcha, todo mundo exceto alguns medrosos, e,
esse cartaz, não o arrancávamos nunca.

Íamos colar nossos cartazes e, uma vez feito o trabalho, voltávamos a
passar na frente deles para vê"-los: eram humildes, apesar da violência
das palavras de ordem e, quanto a mim, não estava seguro de estar no meu
direito de colar aqueles cartazes nos muros: qual era, no fundo, essa
briga que se arrastava tanto? Os comunistas estavam na cidade como um
caroço numa fruta e eu era de carne, não da mesma massa da qual são
feitos os comunistas.

É impossível deixar a sua tribo, ela é muito ampla, está por toda parte.
Está na luz acima do canal, sobre o canal. Homens e mulheres se cansam
nas barcas ou no caminho, para puxá"-las, os braços balançando, um passo
atrás do outro, curvados para baixo: uma mulher com cabelos grisalhos,
longos, que pendem, uma mulher com cabelos grisalhos que puxa sozinha
uma barca. Estou com essa mulher ou com a luz que se choca contra o
canal? Trata"-se mesmo de um ``\versal{OU}'': estou com a mulher, eu desço para
puxar com ela porque não se deixa uma velha puxar sozinha, mas não verei
mais a luz; deve"-se escolher: a velha ou a luz; azar!, a luz, escolho a
luz: por acaso um homem ``escolhe'' o ar? Necessita dele.

A tribo está por toda parte:~está na imensa biblioteca universitária na
qual homens, um ruivo, um negro alto, um velho, estavam ali para nos
adiantar os livros, para nós, os moleques: homens feitos, que haviam
participado da guerra e não sei quê, tudo o que alguns homens têm que
fazer antes de se tornarem visivelmente feitos e tudo o que tinham que
fazer. Nesse caso, tratava"-se de servir a vocês os livros que vocês
tinham que ler: estavam ali como garçonetes num restaurante, e isso
acabava com qualquer vontade de ler. Usavam camisas cinzas. Um deve ter
recebido um estilhaço de granada na cara, um dia, entre 1914 e 1918;
parecia que ele usava uma máscara desanimada: ele lhe estendia o livro
pedido como um mendigo estende o folheto do horóscopo ou o pacote de
envelopes que ele finge estar vendendo. E era nessa biblioteca que
alguém teria que aprender?

A tribo estava nos bares nos quais todos os que sentiam que sobravam no
mundo se faziam companhia durante todo o dia. A tribo estava na
literatura, outra luz que eu devorava com os olhos: a literatura me
falava de tudo. Estava acima da vida dos homens, como a luz acima do
canal e, ao ter que escolher entre viver ou ler, tínhamos que escolher
ler, já que não sabíamos o que viver.

A tribo estava então por toda parte. Para abandoná"-la, era preciso ir
longe.

Lá. O asilo.\footnote{Manicômio.} 1940. A guerra. 1941. O asilo ao longo
de anos. Havia encontrado um ofício: educador. 1943: escrevo
\emph{Semente de crápula}, o qual, seis anos mais tarde, me proponho
criticar. \emph{Semente de crápula} foi lido por alguns milhares de
educadoras e educadores em busca de seu ofício: cinquenta páginas, todas
nutridas pelo leite da tribo pequeno"-burguesa.

Entretanto, trabalhei muito
neste asilo: havia participado da guerra e de tudo o que um homem deve
fazer para parecer um homem feito. Havia um, entre os grandes
caracteropatas, um entre os cem dos que eu não me separava, que andava,
inclinado para frente, um passo depois do outro; contudo, ele não puxava
nada além de si mesmo, e eu estava com ele, no mesmo caminho de terra,
corredor de asilo; ajudava"-o e via, \versal{AINDA ASSIM}, a plena luz do céu.

Mas a tribo é potente: ela era, no próprio crânio de meu estranho
companheiro, nos seus gestos difíceis e fragmentados de
falso"-trabalhador.

Quando tomei nota das frases de \emph{Semente de crápula}, para quem
anotei? Para ninguém: e isso também era coisa da tribo: uma maneira de
fazer bolhas na luz que banha os canais e os campos e as praias enquanto
o homem, no fundo, fica à sua própria pena.

E partir, como muitos outros tentaram, partir para o Taiti, para o
Evereste ou para os Asilos, partir para deixar sua tribo, isso é deixar
o homem: ainda que se parta para viver com os mais desgraçados, os
tortos, os que babam, os malvados, os mancos: um a mais, um a menos, os
mancos estão pouco se fodendo, ou mais ou menos.

Não se deve abandonar os outros. Não se trata de abandonar sua ampla e
sutil tribo pequeno"-burguesa. Deve"-se estar com esses que trabalham para
convencer de que chegou a hora de mudar de ponto de vista. Deve"-se
dizer. Deve"-se cortar, com os dentes se necessário, as amarras que
prendem suas barracas nos lugares desgastados, em plena fome moral.

Desta vez, pelo menos, sei para quem escrevo: para o leitor de
\emph{Semente de crápula} editado em 1945.

Convido"-os, este ano, para perseguirem nas frases das quais gostaram
naquele momento, o hábil pequeno"-burguês educador, que, manejando como
um malabarista suas próprias contradições, elaborou"-as em paradoxos\ldots{}
et cætera\ldots{} Charlatão de boa vontade.
